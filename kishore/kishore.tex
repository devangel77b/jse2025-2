\documentclass[12pt,conference,onecolumn]{IEEEtran}

\usepackage[hidelinks]{hyperref}

\title{Low cost intravenous bag weight monitor}
\author{%
\IEEEauthorblockN{Nitika Kishore}\IEEEauthorblockA{Science \& Engineering\\Manalapan High School\\Englishtown, NJ\\\href{mailto:426nkishore@frhsd.com}{426nkishore@frhsd.com}}\and
\IEEEauthorblockN{Srilekha Dantu}\IEEEauthorblockA{Science \& Engineering\\Manalapan High School\\Englishtown, NJ\\\href{mailto:426sdantu@frhsd.com}{426sdantu@frhsd.com}}}
\date{January 28, 2026}

\newcommand{\keywords}{biomedical engineering, ardunio, intravenous, IV}

\usepackage{hyperref}
\makeatletter
\AtBeginDocument{
\hypersetup{%
pdftitle={\@title},
pdfauthor={Nitika Kishore and Srilekha Dantu},
pdfkeywords={\keywords}}}
\makeatother

\begin{document}
\maketitle 

\begin{abstract}
Intravenous (IV) drips are widely used in clinical settings to maintain patient hydration and prevent complications. Although electronic IV pumps provide continuous monitoring, their high cost limits implementation in low-resource hospitals. Our low cost IV bag weight monitor device offers an affordable alternative by detecting when an IV bag reaches a low volume threshold. The device uses a single-point load cell to measure bag weight and triggers a visual LED indicator and audible buzzer when the weight falls below a preset limit. The current prototype uses a simple shelf-style structure to house the Arduino-based circuitry and display the alarm components. Registered nurses have reviewed the current prototype, deeming it a feasible and efficient alternative to more expensive IV pumps. Future work will integrate a microcontroller and a compact 3D-printed encapsulation to improve durability, reduce size, and improve overall usability.
\end{abstract}

\begin{IEEEkeywords}
\keywords
\end{IEEEkeywords}

\end{document}
