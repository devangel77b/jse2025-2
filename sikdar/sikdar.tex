\documentclass[12pt,conference,onecolumn]{IEEEtran}

\title{Artificial Intelligence (AI)-based adaptive traffic control system for intersection saturation flow rate and congestion optimization: Internship at AT\&T spell out CAHSAA}
\author{%
\IEEEauthorblockN{Adrit Sikdar}\IEEEauthorblockA{Science \& Engineering\\Manalapan High School\\Englishtown, NJ\\426asikdar@frhsd.com}\and
\IEEEauthorblockN{Samay Prabhu}\IEEEauthorblockA{Science \& Engineering\\Manalapan High School\\Englishtown, NJ\\426sprabhu@frhsd.com}}

\date{January 28, 2026}

\newcommand{\keywords}{AT\&T, internship, convolutional neural network, artificial intelligence, reinforcement learning, intelligent transportation systems, traffic control, adaptive traffic control, Deep Q network, machine vision}

\usepackage{hyperref}
\makeatletter
\AtBeginDocument{
\hypersetup{%
pdftitle={\@title},
pdfauthor={Adrit Sikdar and Samay Prabhu},
pdfkeywords={\keywords}}}
\makeatother

\begin{document}
\maketitle 

\begin{abstract}
Intelligent transportation systems have been a topic of major interest throughout the past few decades. Modern adaptive traffic control systems (ATCS) have had consistent development, with individual logic frameworks and architectures developed in order to improve intersection efficiency through adaptive light phase timing. This project focuses on the optimization of saturation flow rate (SFR) for intersections with an ATCS. The model includes a pipeline with a Convolutional Neural Network (CNN) backbone structure for object detection and classification, alongside a Deep Q Network (DQN) based traffic signal phase optimization algorithm. Object detection is implemented at all points of interest at a given intersection, providing congestion data across all lanes used to maximize SFR. This congestion data is processed into a DQN based optimizer algorithm, which dynamically updates phase timings for traffic light states in real time. The system provides an increase in average SFR and reduced congestion within a variety of intersections in ideal conditions over classical fixed timing mechanisms currently in use. This project provides both economical and environmental benefits, alongside reducing average travel time for both civilian and government vehicles.
\end{abstract}

\begin{IEEEkeywords}
\keywords
\end{IEEEkeywords}

\end{document}
