\documentclass[12pt,conference,onecolumn]{IEEEtran}

\usepackage[hidelinks]{hyperref}

\title{Math-focused screenreader for accessibility: Internship at IEEE}
\author{%
\IEEEauthorblockN{Aryanna Cetrulo}\IEEEauthorblockA{Science \& Engineering\\Manalapan High School\\Englishtown, NJ\\\href{mailto:426acetrulo@frhsd.com}{426acetrulo@frhsd.com}}\and
\IEEEauthorblockN{Anika Tokala}\IEEEauthorblockA{Science \& Engineering\\Manalapan High School\\Englishtown, NJ\\\href{mailto:426atokala@frhsd.com}{426atokala@frhsd.com}}\and
\IEEEauthorblockN{K August}\IEEEauthorblockA{\href{mailto:kit.august@gmail.com}{kit.august@gmail.com}}}

\date{January 28, 2026}

\newcommand{\keywords}{assistive technology, mathematics, screen reader, blind, visually impaired, low vision, Americans with Disabilities Act, ADA, access, mathematical notation, inclusive design, internship, IEEE}

\usepackage{hyperref}
\makeatletter
\AtBeginDocument{
\hypersetup{%
pdftitle={\@title},
pdfauthor={Aryanna Cetrulo, Anika Tokala, and K August},
pdfkeywords={\keywords}}}
\makeatother

\begin{document}
\maketitle 

\begin{abstract}
Mathematics relies heavily on symbolic notation, spatial relationships, and non-linear structures that challenge traditional text-to-speech systems. While screen readers have improved accessibility for reading text, their performance for mathematics remains insufficient due to inconsistent formatting standards, unclear interpretation of mathematical meaning, and a lack of audio representation for visually intuitive math layouts and structures. We were tasked with evaluating current screen reader technologies and developing ideas to improve mathematical accessibility for visually impaired users. Our research involved testing existing screen readers with mathematical symbols, conducting interviews with visually impaired individuals about their experiences accessing mathematical content, and identifying accessibility features that should be included into our website from the initial design phase. 
\end{abstract}

\begin{IEEEkeywords}
\keywords
\end{IEEEkeywords}

\end{document}
