\documentclass[12pt,conference,onecolumn]{IEEEtran}

\title{Hybrid BiLSTM machine learning with RNAfold-based thermodynamic modeling for RNA secondary structure prediction}
\author{%
\IEEEauthorblockN{Jophy Lin}\IEEEauthorblockA{Science \& Engineering\\Manalapan High School\\Englishtown, NJ\\126jlin@frhsd.com}}
\date{January 28, 2026}

\newcommand{\keywords}{deep learning, RNA, thermodynamic modeling, ribonucleic acid, secondary structure, BiLSTM neural network, Monte Carlo, machine learning, hybrid, RNAfold}

\usepackage{hyperref}
\makeatletter
\AtBeginDocument{
\hypersetup{%
pdftitle={\@title},
pdfauthor={Jophy Lin},
pdfkeywords={\keywords}}}
\makeatother

\begin{document}
\maketitle 

\begin{abstract}
RNA secondary structure plays an important role in various biological fields, such as gene regulation, catalysis, and RNA–protein interactions, yet accurate prediction is challenging due to long-range base-pairing and structural complexity. This work presents a hybrid framework that integrates RNAfold's thermodynamic modeling with machine learning to improve RNA secondary structure prediction and interpretability. Using a balanced dataset of approximately 9,700 RNA sequences spanning multiple RNA families ($\approx$ 4.09 million nucleotides), traditional thermodynamic prediction with RNAfold was evaluated alongside a bidirectional long short-term memory (BiLSTM) neural network trained for base-wise pairing prediction. While RNAfold achieved a baseline test accuracy of 0.77, the BiLSTM reached 0.91 accuracy. Building on these results, several hybrid approaches were developed that selectively combine RNAfold and neural predictions, including a base-wise selector, a sequence-level meta-learner, and a Monte Carlo (MC) dropout uncertainty method. The best-performing hybrid model, the MC dropout uncertainty method, achieved a test accuracy of 0.917, outperforming both standalone approaches. This framework has been deployed through an interactive web interface, enabling users to input RNA sequences and compare prediction methods in real-time. To enhance interpretability, predicted structures are converted from dot-bracket notation into annotated visual diagrams depicting the corresponding secondary structure motifs. This study demonstrates that hybrid modeling with uncertainty-aware selection can improve RNA secondary structure prediction while maintaining accessibility and interpretability for later biological analysis.
\end{abstract}

\begin{IEEEkeywords}
\keywords
\end{IEEEkeywords}

\end{document}
