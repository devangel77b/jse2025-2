\documentclass[12pt,conference,onecolumn]{IEEEtran}

\title{Origami-engineered biopolymer patch for localized nanoparticle delivery in pediatric skin cancer and post-tumor removal therapy}
\author{%
\IEEEauthorblockN{Dia Avalur}\IEEEauthorblockA{Science \& Engineering\\Manalapan High School\\Englishtown, NJ\\426davalur@frhsd.com}}
\date{January 28, 2026}

\newcommand{\keywords}{origami, drug delivery, therapy, diffusion kinetics, pediatric oncology, pH, biomedical engineering}

\usepackage{hyperref}
\makeatletter
\AtBeginDocument{
\hypersetup{%
pdftitle={\@title},
pdfauthor={Dia Avalur},
pdfkeywords={\keywords}}}
\makeatother

\begin{document}
\maketitle 

\begin{abstract}
Children recovering from cancer treatments often develop painful skin lesions that require frequent topical medication, which can be uncomfortable and inconsistent. This study presents a moisture-triggered sodium-alginate patch that folds in an origami-like pattern to deliver tannic acid, a natural antioxidant, in a controlled and localized way. Sodium alginate is an inexpensive, biocompatible hydrogel material, and tannic acid, found naturally in green tea, is gentle and soothing on irritated skin, making the combination ideal for pediatric care. The patch’s design and crosslinking density were optimized to regulate swelling and diffusion under both physiological (pH 7.4) and wound-like (pH 5.5) conditions. Flat and origami geometries were modeled using Fick’s Law of diffusion to study how pH and crosslinking influence release behavior. Data collected over nine time points (0--24 h) were analyzed using ANOVA and kinetic model fitting. The origami design released nearly twice as much tannic acid under physiological conditions as the flat, highly crosslinked design in acidic conditions, with cumulative release values of 78--80\% and 39--41\%, respectively. Increased surface area and moisture exposure improved diffusion release rates, while tighter crosslinking reduced pore size and slowed release rates. Overall, this work demonstrates that a soft, moisture-responsive hydrogel patch can provide a gentle, low-cost, and effective method for targeted treatment in pediatric skin cancer recovery. Future work will focus on creating physical prototypes and validating diffusion performance experimentally.
\end{abstract}

\begin{IEEEkeywords}
\keywords
\end{IEEEkeywords}

\end{document}
