\documentclass[12pt,conference,onecolumn]{IEEEtran}

\title{Si vis pacem, para ballista}
\author{%
\IEEEauthorblockN{Ryan Leung}\IEEEauthorblockA{Science \& Engineering\\Manalapan High School\\Englishtown, NJ\\426rleung@frhsd.com}\and
\IEEEauthorblockN{Anton Lavrenov}\IEEEauthorblockA{Science \& Engineering\\Manalapan High School\\Englishtown, NJ\\426alavrenov@frhsd.com}}
\date{January 28, 2026}

\newcommand{\keywords}{ballista, siege weapon, siege engine, ancient Greece, ancient Rome, antiquity, Middle Ages, history}

\usepackage{hyperref}
\makeatletter
\AtBeginDocument{
\hypersetup{%
pdftitle={\@title},
pdfauthor={Ryan Leung and Anton Lavrenov},
pdfkeywords={\keywords}}}
\makeatother

\begin{document}
\maketitle 

\begin{abstract}
A ballista is an ancient siege weapon, resembling a large crossbow, and typically used during the time of the Ancient Greeks and Romans to hurl large projectiles. Instead of storing energy in the bending of its limbs like a crossbow, ballistae are powered by two twisted bundles of rope (the skein). The arms of the ballista are suspended within the skein, which are wound tight enough to allow the arms to spring forward after they are released from a loaded position. The firing power of a ballista is mainly based on how tightly wound the skein is, and thus we tested a multitude of ropes using a testing rig consisting of a dynamometer and a chain hoist. The rope would be attached to the aforementioned parts, which would allow us to subsequently find the energy stored in the rope. Throughout this project, we tested different types of rope, modeled a ballista based on Greek and Roman works, and then finally built a final design.  Give indication of specific time period you are most interested in and give more of what was accomplished. 
\end{abstract}

\begin{IEEEkeywords}
\keywords
\end{IEEEkeywords}

\end{document}
