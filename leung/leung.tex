\documentclass[12pt,conference,onecolumn]{IEEEtran}

\title{Si vis pacem, para ballista}
\author{%
\IEEEauthorblockN{Ryan Leung}\IEEEauthorblockA{Science \& Engineering\\Manalapan High School\\Englishtown, NJ\\426rleung@frhsd.com}\and
\IEEEauthorblockN{Anton Lavrenov}\IEEEauthorblockA{Science \& Engineering\\Manalapan High School\\Englishtown, NJ\\426alavrenov@frhsd.com}}
\date{January 28, 2026}

\newcommand{\keywords}{ballista, siege weapon, siege engine, ancient Greece, ancient Rome, antiquity, Middle Ages, history, E W Marsden, Vitruvius, Hero of Alexandia, Biton of Macedonia, tensile test, energy, projectile motion, elasticity, elastic energy storage, experimental archaeology, medieval reenactors, Society for Creative Anchronism}

\usepackage{hyperref}
\makeatletter
\AtBeginDocument{
\hypersetup{%
pdftitle={\@title},
pdfauthor={Ryan Leung and Anton Lavrenov},
pdfkeywords={\keywords}}}
\makeatother

\begin{document}
\maketitle 

\begin{abstract}
A ballista is an ancient siege weapon, used from 400 BCE to 400 CE, resembling a large crossbow, and typically used during the time of the Ancient Greeks and Romans to hurl large projectiles. They were typically used for long range, high precision sniping.  Instead of storing energy in the bending of its limbs like a crossbow, ballistae are powered by two twisted bundles of rope (the skeins). The arms of the ballista are suspended within the skeins, which are wound tight enough to allow the arms to spring forward after they are released from a loaded position. The firing force of a ballista and energy transmitted to the projectile are mainly based on how tightly wound the skeins are, and thus we tested a multitude of ropes using a tensile testing rig consisting of a dynamometer and a chain hoist. The rope would be attached to the aforementioned parts, which would allow us to subsequently find the energy stored in the rope. Throughout this project, we were able to strengthen our CAD and woodworking abilities, whilst also learning about weapons of the time period, specifically through literature like Marsden's \emph{Greek and Roman Artillery: Technical Treatises}, primary sources such as Vitruvius, Hero of Alexandria, and Biton of Macedonia, and contact with medieval reenactors and experimental archaeology enthusiasts from the Society for Creative Anachronism.  
\end{abstract}

\begin{IEEEkeywords}
\keywords
\end{IEEEkeywords}

\end{document}
