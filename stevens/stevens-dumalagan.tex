\documentclass[12pt]{article}
\usepackage[margin=1in]{geometry}

\begin{document}

\begin{center}
\textbf{\Large Analysis Of A Golf Swing}
\end{center}

\vspace{1em}

\begin{center}
\begin{tabular}{p{2.8in} p{2.8in}}
\centering
Clay Stevens\\
Science and Engineering\\
Manalapan High School\\
Englishtown, NJ\\
426cstevens@frhsd.com
&
\centering
Aidan Dumalagan\\
Science and Engineering\\
Manalapan High School\\
Englishtown, NJ\\
426adumalagan@frhsd.com
\end{tabular}
\end{center}

\vspace{1.5em}

\begin{center}
\textbf{Abstract}
\end{center}

The project focuses on optimizing the golf swing through a physics-based analysis of biomechanics and kinematics. We began studying the fundamental foundations of the golf swing, including the double-pendulum model, which simplifies the motion by transferring energy from the arms (first pendulum) to the wrists and club (second pendulum) to achieve maximum power at impact. To establish a baseline, we recorded our initial swings and compared them to professional golfer Rory McIlroy, analyzing differences in joint angles, limb displacement, and timing using software including Factorial Biomechanics and TrajectoWare. As the comparison highlighted areas for improvement, specifically in swing sequencing and efficient transfers of energy, we applied these targeted adjustments to our swing, allowing for a measurable improvement in efficiency and power. Our final analysis provides a comprehensive, physics-based perspective on swing optimization along with new videos of our new and improved swings.

\end{document}

