\documentclass[12pt,conference,onecolumn]{IEEEtran}

\usepackage[hidelinks]{hyperref}

\title{Analysis of a golf swing}
\author{%
\IEEEauthorblockN{Clay Stevens}\IEEEauthorblockA{Science \& Engineering\\Manalapan High School\\Englishtown, NJ\\\href{mailto:426cstevens@frhsd.com}{426cstevens@frhsd.com}}\and
\IEEEauthorblockN{Aidan Dumalagan}\IEEEauthorblockA{Science \& Engineering\\Manalapan High School\\Englishtown, NJ\\\href{mailto:426adumalagan@frhsd.com}{426adumalagan@frhsd.com}}}
\date{January 28, 2026}

\newcommand{\keywords}{biomechanics, physics, golf, swing kinematics, double-pendulum model, Rory McIlroy, Factorial Biomechanics, TrajectoWare}

\usepackage{hyperref}
\makeatletter
\AtBeginDocument{
\hypersetup{%
pdftitle={\@title},
pdfauthor={Clay Stevens and Aidan Dumalagan},
pdfkeywords={\keywords}}}
\makeatother

\begin{document}
\maketitle 

\begin{abstract}
The project focuses on optimizing the golf swing through a physics-based analysis of biomechanics and kinematics. We began studying the fundamental foundations of the golf swing, including the double-pendulum model, which simplifies the motion by transferring energy from the arms (first pendulum) to the wrists and club (second pendulum) to achieve maximum power at impact. To establish a baseline, we recorded our initial swings and compared them to professional golfer Rory McIlroy, analyzing differences in joint angles, limb displacement, and timing using software including Factorial Biomechanics and TrajectoWare. As the comparison highlighted areas for improvement, specifically in swing sequencing and efficient transfers of energy, we applied these targeted adjustments to our swing, allowing for a measurable improvement in efficiency and power. Our final analysis provides a comprehensive, physics-based perspective on optimizing swing output and efficiency, along with new videos of our new and improved swings.
\end{abstract}

\begin{IEEEkeywords}
\keywords
\end{IEEEkeywords}

\end{document}
