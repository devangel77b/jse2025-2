\documentclass[12pt,conference,onecolumn]{IEEEtran}

\title{Modeling swarm spacecrafts in Python: Internship at Girl in Space Club}
\author{%
\IEEEauthorblockN{Jake Chin}\IEEEauthorblockA{Science \& Engineering\\Manalapan High School\\Englishtown, NJ\\426jchin@frhsd.com}\and
\IEEEauthorblockN{Aleksandra Guimaraes}\IEEEauthorblockA{Science \& Engineering\\Manalapan High School\\Englishtown, NJ\\426aguimaraes@frhsd.com}\and
\IEEEauthorblockN{Sabrina Thompson}\IEEEauthorblockA{Girl in Space Club\\sabrina.thompson@girlinspaceclub.com}}
\date{January 28, 2026}

\newcommand{\keywords}{Girl in Space Club, web design, STEM education, orbital mechanics, Python, swarm control, satellite, simulation, AI, artificial intelligence, swarm autonomy, relative motion, internship}

\usepackage{hyperref}
\makeatletter
\AtBeginDocument{
\hypersetup{%
pdftitle={\@title},
pdfauthor={Jake Chin, Aleksandra Guimaraes, and Sabrina Thompson},
pdfkeywords={\keywords}}}
\makeatother

\begin{document}
\maketitle 

\begin{abstract}
Girl in Space Club is a creative technology and human-systems design lab. They design, prototype, and commercialize equipment, tools, and experiences for people navigating future worlds—on Earth and beyond. It was founded by NASA engineer Sabrina Thompson.  As part of our mission, we were tasked with creating a full-stack simulation tool that models autonomous spacecraft swarms powered by AI and orbital mechanics. This will be used to demonstrate AI-driven decision-making for a swarm of spacecraft executing orbital maneuvers in response to goals, threats, or mission objectives for both educational and technological purposes. During our presentation, we will present (1) our research involving swarm autonomy rules, their relative motion, and orbital mechanics; (2) how we translated this research into visual representations; and (3) what we plan to implement into our interface in the future for the backend, as well as any issues we encountered.
\end{abstract}

\begin{IEEEkeywords}
\keywords
\end{IEEEkeywords}

\end{document}
