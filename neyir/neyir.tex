\documentclass[12pt,conference,onecolumn]{IEEEtran}

\title{Compact integrated drone payload delivery system}
\author{%
\IEEEauthorblockN{Timur Neyir}\IEEEauthorblockA{Science \& Engineering\\Manalapan High School\\Englishtown, NJ\\426tneyir@frhsd.com}\and
\IEEEauthorblockN{Sameera Patil}\IEEEauthorblockA{Science \& Engineering\\Manalapan High School\\Englishtown, NJ\\426spatil@frhsd.com}}
\date{January 28, 2026}

\newcommand{\keywords}{drone, flight, mechanics, payload, winch, quadrotor}

\usepackage{hyperref}
\makeatletter
\AtBeginDocument{
\hypersetup{%
pdftitle={\@title},
pdfauthor={Timur Neyir and Sameera Patil},
pdfkeywords={\keywords}}}
\makeatother

\begin{document}
\maketitle 

\begin{abstract}
Our project is centered on designing and developing an RC drone that supports a payload system. Our project aims at developing a working prototype capable of supporting stable flight and delivering precision with respect to payload. Our approach towards achieving this is through designing a light support mechanism that is ideal and sized for a small base of a quadcopter design. This is through a winch mechanism that is capable of controlled extension and retraction. This has been modeled using CAD design software, and our design has been centered on reducing its weight and having ideal strength that is compatible with a 6-inch drone frame. Alongside our design of the mechanism, we have also mapped out the electric design necessary for our mechanism to operate effectively. This entails a DC motor that supports our mechanism, receiver-transmitter, Arduino, and LiPo battery that is not compatible with drone operations. Our design has been informed by factors such as its capacity to operate with a low payload mass, its size corresponding to that of a drone, as well as its capability to perform its operations while sustaining stable flight as it extends and retracts. The project is actively progressing through continuous research and development including assembly, 3D printing, and testing. It’s through this work that we aim to show a feasible design of a compact system that also addresses the challenges associated with drones of this size.
\end{abstract}

\begin{IEEEkeywords}
\keywords
\end{IEEEkeywords}

\end{document}
