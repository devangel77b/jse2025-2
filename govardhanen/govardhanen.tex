\documentclass[12pt,conference,onecolumn]{IEEEtran}

\usepackage[hidelinks]{hyperref}

\title{A computational study of chaos in simplified microbial population models}
\author{%
\IEEEauthorblockN{Vasudevan Govardhanen}\IEEEauthorblockA{Science \& Engineering\\Manalapan High School\\Englishtown, NJ\\\href{mailto:426vgovardhanen@frhsd.com}{426govardhanen@frhsd.com}}\and
\IEEEauthorblockN{Siddharth Kedharnath}\IEEEauthorblockA{Science \& Engineering\\Manalapan High School\\Englishtown, NJ\\\href{mailto:426skedharnath@frhsd.com}{426skedharnath@frhsd.com}}\and
\IEEEauthorblockN{Brady Gorelczenko}\IEEEauthorblockA{Science \& Engineering\\Manalapan High School\\Englishtown, NJ\\\href{mailto:426bgorelczenko@frhsd.com}{426bgorelczenko@frhsd.com}}}

\date{January 28, 2026}

\newcommand{\keywords}{dynamical systems, chaos theory, partial differential equation, PDE, diffusion, Lotka-Volterra, simulation, stability, Lyapunov exponent, microbes, ecology, population ecology}

\usepackage{hyperref}
\makeatletter
\AtBeginDocument{
\hypersetup{%
pdftitle={\@title},
pdfauthor={Vasudevan Govardhanen, Siddharth Kedharnath, and Brady Gorelczenko},
pdfkeywords={\keywords}}}
\makeatother

\begin{document}
\maketitle 

\begin{abstract}
This project investigates the emergence of nonlinear and chaotic dynamics in microbial population models using purely computational and mathematical approaches. Reaction diffusion systems, logistic growth equations, and Lotka-Volterra style interactions are implemented to study how diffusion, growth rates, carrying capacity, and initial conditions shape population behavior over time. By systematically varying parameters and introducing small perturbations, the model explores sensitivity to initial conditions, bifurcations, and transitions between stable equilibria, oscillations, and irregular dynamics. Numerical simulations generate time series data, phase space plots, and bifurcation diagrams, with chaos quantified using metrics such as Lyapunov exponents. While the current framework is intentionally simplified and limited in its biological realism, it functions as a controlled environment for isolating mathematical mechanisms that produce complex behavior.
\end{abstract}

\begin{IEEEkeywords}
\keywords
\end{IEEEkeywords}

\end{document}
